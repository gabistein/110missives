\documentclass[a4paper]{article}

%% Language and font encodings
\usepackage[english]{babel}
\usepackage[utf8x]{inputenc}
\usepackage[T1]{fontenc}
\usepackage{titletoc}
%% Sets page size and margins
\usepackage[a4paper,top=3cm,bottom=2cm,left=3cm,right=3cm,marginparwidth=1.75cm]{geometry}

%% Useful packages
\usepackage{amsmath}
\usepackage{graphicx}
\usepackage[colorinlistoftodos]{todonotes}
\usepackage[colorlinks=true, allcolors=blue]{hyperref}

\title{Team 1101}
\author{UTA Missives}

\begin{document}
\maketitle
\clearpage
\startcontents[mainsections]
\printcontents[mainsections]{l}{1}{\section*{Table Of Contents}\setcounter{tocdepth}{2}}
\clearpage
\section{About this document}

This document is intended to outline the various responsibilities of undergraduate teaching assistants for Team 1101 within the Department of Computer Science at UNC and provide professional development information for entering industry or academia. 

\subsection{Contributors}

This document is a living document and is reviewed and enhanced by the lead-TAs each year. It has been updated for the current year by Gabi Stein. 

\section{Introduction to the UTA Program}

Team 1101 is committed to providing students the opportunity to assist in administering and developing COMP110 and COMP101. By having a team of 50-60 UTAs, these courses are able to be personal and meaningful to all students. 

\subsection{Structure}

Team 1101 is composed of 59 UTAs who all work scheduled office hours. Outside of office hours, 1-2 lead TAs are designated to run different groups to enhance the course. These groups include: video content, grading/schedule generation, tutoring/review sessions, worksheets, problem sets,  operations/logistics, and website management. The chain of command for a topic is typically: professor->lead TA->UTA team. However, this may differ depending on the matter. 
\subsection{Relevance}

Becoming a UTA on Team 1101 can be the most important aspect of an education in computer science at UNC. Our dedicated team of TAs allows COMP110 and COMP101 to assign more interesting coursework, offer more detailed and quicker feedback, and hold more office hours. The TAs themselves benefit from working closely with a professor, developing the types of skills sought after by employers and graduate schools, and having the great satisfaction of helping their peers learn the material. Team 1101 shares a culture of empathy for all students, desire to learn cutting edge programming, and accountability.

\subsection{Candidate eligibility}
Everyone who has taken COMP110 is eligible to apply. We look for the following in new team members: 
\begin{itemize}
\item Skilled communicators with empathy. 
\item Dependable team players
\item Intelligent, driven, creative peers
\item Optimists who enjoy having fun 
\end{itemize}
\begin{flushleft}
All members of the team found programming in COMP110, too. Becoming a junior UTA while you're simultaneously in COMP401 is an amazing catalyst for strengthening your skills and knowledge.
\end{flushleft}
\begin{flushleft}
COMP110 UTAs are hired at two levels of weekly commitment. Typically, new team members are hired for 5 hours per week. Some on the team opt for 10 hours per week when funding is available. Additionally, Sundays are our team meetings where we regroup and prep for the week ahead.
\end{flushleft}
\subsection{UTA duties}
Each semester TA duties differ based on new goals Kris might have for the course. However, some basic responsibilities are expected of all TAs.
\begin{flushleft}
Most, if not all, UTAs are expected to:
\end{flushleft}
\begin{itemize}
\item hold office hours during which students may approach them with questions regarding assignments or material covered in class; 
\item grade assignments (written and/or programming), exams,etc. according to standards and material established by the professor; 
\item respond in emails sent to from their assigned students; 

\item adhere to all university and department policies and procedures; including but not limited to those described in this document; 
\item attend weekly staff meetings; and
\item conduct themselves in a professional manner at all times, both inside and outside of Sitterson. 
\end{itemize}
\subsection{Additional tasks}
Team 1101 has sub-teams that welcome TAs to assist in the following:
\begin{itemize}
\item hold interactive grading sessions for design checks and programming assignments; 
\item build hand-written worksheets
\item prepare and present help sessions for projects and upcoming exams
\item build new problem sets
\item manage the course website
\item produce video content to support material covered in lecture
\item plan and coordinate a diversity panel and hackathon for the introductory computer science community
\end{itemize}
\subsection{Tasks not handled by UTAs}
Some tasks are outside of the scope of a UTA's responsibility. For example, UTAs do not:
\begin{itemize}
\item assign final grades and enter them in Connect Carolina. 
\item teach courses, unless you are Jeffrey 
\item confront students about possible violations of the academic code, unless you see collaboration in the Sitterson lobby; 
\item prosecute cases of alleged academic misconduct; or 
\item grant extensions or Incompletes to students 
\end{itemize}
\subsection{UTA Leave Policy}
During the semester, UTAs may find that they need to take a leave from their duties for
health reasons. A leave of absence entails a hiatus of some or all of TAs responsibilities. If 
you feel, at any point during the semester, that you would benefit from a leave of absence, contact Kris. 
\subsection{Substandard performance}
UTAs who are unable to perform the basic functions of their job satisfactorily will be dealt
with on a case-by-case basis. In extreme situations, a UTA may be dismissed for poor
performance or for engaging in unprofessional behavior. 
\subsection{Hours}
UTAs are either assigned to work 5 hours or 10 hours per week. A portion or all of these hours are dedicated to office hours. In general, you should only be helping one student at a time on duty, and other students seeking your expertise should wait outside in the lobby. 
\subsection{Location and timing}
Office hours are held in SN008. Throw away any trash that may have accumulated during your hours and erase any white boards you may have written on. Keep the door to the room you are in open. If you shift around chairs, put the room back together before leaving. Also, if you are the last TA to leave the room at night, turn off the lights and close the door after you. 
\begin{flushleft}
It is your responsibility to arrive at your hours on time and to be appropriately prepared. Do
not be late, do not wander away from your post, and do not cancel your hours at the last
minute. If you know in advance that you will be unable to attend your hours, make every
effort to trade times with another TA. If no other TA is available, then contact the lead TA
to reschedule your hours for another time that week. 
\end{flushleft}
\subsection{Waitlists}
The waitlist for COMP110/COMP101 is kept on Course.Care. Students must create a ticket on course.care to be served. 
\begin{flushleft}
You may need to cutoff your waitlist if it is still growing towards the end of your shift and you are the last TA on hours. You may also want to slack your fellow TAs requesting backup if the list gets out of hand. 
\end{flushleft}
\begin{flushleft}
Do not allow students to ignore Course.Care, show them how to create a ticket if they do not know. 
When there is a waitlist, there is a 15 minute time-limit on interactions, make sure you are enforcing the limit and set realistic expectations with each student interaction. 
\end{flushleft}
\subsection{Grading complaints}
If a student comes to you on hours to contest a grade, politely turn them to submit a grade request on Gradescope in the time frame provided. If they persist, contact Kris. 
\subsection{Teaching tactics}
Arrive to your hours well-rested, well-fed, and sober. Make sure you are well-versed in the
material that has been covered recently in class. Take a look at whatever assignment is
currently out; if you do not feel confident in your ability to solve the problems or your
memory of the project is fuzzy, go over your old code(if available) or reach out to a fellow UTA. Do not bluff your
way to an answer; if you are not sure, ask another TA, or tell the student you will investigate
and get back to them shortly. Giving bad advice or incorrect information at TA hours is
harmful not only to the student but also to the TA and the course. 
\subsection{Handling problematic students}
Some students under pressure may be emotional when they come to you on hours. If a student seems under extreme duress and it is clear that they will not be able to finish the  project or homework before the deadline, gently suggest they have something to eat, get some sleep, and take a late penalty. Do not offer to give the student an extension or Incomplete; only a professor can award these, and the latter will likely involve a
recommendation by a Dean who will talk to the student about their special circumstances. If
the student asks about an extension, refer them to your professor, but do not comment on
the likelihood of an extension being granted. If a student is particularly aggressive, we
encourage you to direct message Kris on slack.
\begin{flushleft}
Some students will pressure you for more help than you are allowed to offer them. Do not
give them answers or snippets of code. It is not your job to supply your students with the
answers; it is your job to help them figure out the answers for themselves. If a student is
constantly pressuring you for more help than is acceptable, you may wish to bring the
situation up with the shift lead or your professor who can help you talk to the student or talk to
the student themselves about what kind of help should be expected at TA hours.
We strongly discourage discussing course material with students outside of your official
hours. If you are approached by a student with questions about the course while you are not
on hours, you are not obligated to answer them. It is to your benefit and that of your co-TAs
that no TAs provide help outside of hours as students will not expect that any TA will help
them outside of hours and will refrain from making such requests. 
\end{flushleft}

\subsection{Your conduct}
Do not speak ill of your students, co-TAs, HTA(s), professor, or course. Do not make
disparaging comments about other courses, TAs, professors, or the department in general.
As a UTA, students will look to you as an example. You will set the tone that they will use to
describe their experiences within your course and, by extension, the CS department.
\begin{flushleft}
Try to appear approachable while on hours. Be patient with your students. Encourage them
to ask questions or request that you slow down or repeat explanations.
\end{flushleft}
\begin{flushleft}
Consider the message your behavior, tone, and appearance sends to your students.
Everything right down to your desktop background should communicate that you are a
professional, respectful individual. If a student does not reciprocate this respectful behavior,
ask them to continue their discussion with your HTA(s), who will have more experience
with these types of situations and may in turn refer them to your professor. 
\end{flushleft}
\subsection{Contributing to a positive work environment}
Humor should not come at the expense of others’ comfort. Sexist, racist, and homophobic
jokes are never funny. Any joke meant to alienate others is never funny. Be aware of the
comments you make while around students and other TAs, even if you are speaking casually
to friends. Something that may seem funny to you may make your fellow TAs or students
uncomfortable in the course which is unacceptable.
\begin{flushleft}
Harassment is any verbal, written, or physical conduct designed to intimidate, coerce, or
make another person feel uncomfortable. Harassment may include unwelcome advances,
physical touching, or offensive or unwelcome comments regarding a person’s race, gender,
nationality, religion, sexual orientation, age, disability, or appearance. The undergraduate TA
program has a zero-tolerance policy for harassment of any form.
\end{flushleft}
\begin{flushleft}
Though the department encourages the social aspects of working in SN008 and the sense
of camaraderie it can foster, do not let your socializing degenerate to the point where it
makes the lab overly noisy and unpleasant for students who are trying to seek help. 
\end{flushleft}
\subsection{Pay and On-boarding}
UTAs are either paid \$1000 for working 5/week or \$2000 for working 10/week each semester. 
\begin{flushleft}
Students Setup accounts on Slack, comp101.org, Course.Care, and comp110.com websites. Kris adds new Team 1101 members to the Google group. 
\end{flushleft}
\begin{flushleft}
Once user accounts are setup for new members, send them e-mail letting them know they need to take the following actions:
\begin{enumerate}
\item \textbf{FERPA Training.} This is not very engaging but it's incredibly important and also easy. Please complete the FERPA training as soon as possible by following these instructions: http://registrar.unc.edu/academic-services/uncferpa/ferpa-instructions/ -- once you have completed the training, please forward the completion e-mail to jogregor@cs.unc.edu and include in the subject line "FERPA completion for COMP110 LA [your name] [your PID]" 
\item \textbf{Slack account.} You should have received an invite to the team's Slack group. Slack is our team chat room. It has both a nice desktop app and mobile app. Go ahead and sign-up and setup your account with a profile picture.
\item \textbf{Website profile.} You should also have received an invite to the back-end of the COMP110.com website. We'll talk about this more, but if you want to go ahead and setup your profile it'd be great. 
\begin{flushleft} Once you do, you'll show up here: http://comp110.com/team - a couple of good example profiles include: \href{http://comp110.com/team/bio?onyen=zihe}{Helen Qin} -- \href{http://comp110.com/team/bio?onyen=brooksmt}{Brooks Townsend}\end{flushleft}
\begin{enumerate}
\item To edit your profile, log into the back-end of the site \href{http://comp110.com/admin}{here}. 
\item the sidebar, perhaps toward the bottom, you'll see your name and a blank profile picture. Click this and click "My Account". The "Profile" tab is where you can fill in information about yourself. Ignore the "recitation" field for now.
\end{enumerate}
\item \textbf{I-9 identification and a voided check.} I'm repeating this because it's important to remember to bring the documentation needed to get hired by the department. In semesters past folks have needed to have identification materials mailed from home which is a huge pain. \href{https://www.brown.edu/about/administration/human-resources/sites/human-resources/files/acceptable%20documents.pdf}{Check out the doucments here!}
\end{enumerate}
\end{flushleft}
\section{Task Forces within Team 1101}
\subsection{Operations and Logistics} 
\begin{enumerate}
\item Head of Ops-1 Hour Commitment
\begin{itemize}
\item Check in biweekly with your team members and check Asana for updates. 
\item Organize hiring process. 
\item Handle unplanned schedule changes. 
\end{itemize}
\item Email-1 Hour Commitment
\begin{itemize}
\item \hyperref[team_email]{Make sure all emails are replied to for course.}
\end{itemize}
\item Outreach-2 Hour Commitment
\begin{itemize}
\item Pull student emails for outreach on problem sets, \hyperref[quiz_outreach]{quizzes, exams}, and worksheets for both courses.
\end{itemize}
\item Schedule-1 Hour Commitment 
\begin{itemize}
\item Generate in-class help schedule at the beginning of semester and mid-semester. 
\item\hyperref[hour_allocation]{Update hour allocation for team.}
\end{itemize}
\end{enumerate}
\subsection{Hands-On Code}
\begin{enumerate}
\item Head of Grading-5 Hour Commitment
\begin{itemize}
\item Develop grading criteria on Gradescope for Quizzes and worksheets. 
\item Staffing grading parties. 
\item Work with Kris to organize \hyperref[exam_grading]{exam grading.} 
\end{itemize}
\item Problem Set PM-3 Hour Commitment
\begin{itemize}
\item Work with Kris at the beginning of the semester to outline P-Set needs (data scaper, API, general dev)
\item Coordinate with Engineer to make a weekly plan to accomplish. 
\item Help with P-set write-ups on the website
\end{itemize}
\item Problem Set Engineer-3 Hour Commitment
\begin{itemize}
\item Work with PM on weekly Pset dev
\item Generate graders for new problem sets
\end{itemize}
\item Code Reviewer-1 Hour Commitment
\begin{itemize}
\item Run code reviews with fellow TAs on 110 P-Sets to discuss style and how to potentially incorporate style into the 110 curriculum. 
\end{itemize}
\item Code Reviewer-1 Hour Commitment 
\begin{itemize}
\item Run code reviews with fellow TAs on 110 P-Sets to discuss style and how to potentially incorporate style into the 110 curriculum. 
\end{itemize}
\end{enumerate}
\subsection{Outside Events}
\begin{enumerate}
\item 2 Hack110 Co-Planners-1-2 Hour Commitment
\begin{itemize}
\item \hyperref[hackathon]{Plan HACK110}
\end{itemize}
\item 1 Diversity Panel Planner-1-2 Hour Commitment 
\begin{itemize}
\item Plan Diversity Panel
\end{itemize}
\item 2 General Planner Committee Members-1 Hour Commitment
\begin{itemize}
\item Help with planning of diversity panel and hackathon
\end{itemize}
\item Team Event Planner-1Hour Commitment
\begin{itemize}
\item Develop calendar of outside events. 
\item Coordinate events to have drivers and reminders of time and date and other logistics as needed. 
\end{itemize}
\end{enumerate}
\subsection{Tutoring and Review Sessions}
\subsubsection{Goals}
\begin{enumerate}
\item Weekly 110 Review
\item Weekly 101 Review
\item Midterm Reviews
\item Weekly 101 tutoring sessions
\item Weekly 110 tutoring sessions
\item Managing and understanding how to improve tutoring and review sessions to accommodate course pace. 
\item Staffing tutoring. 
\end{enumerate}
\subsubsection{Roles}
\begin{enumerate}
\item Head of review sessions-2 Hour Commitment
\item Head of tutoring-2 Hour Commitment
\item Content creator 101 review sessions-1 Hour Commitment
\item Content creator 110 review sessions-1 Hour Commitment
\end{enumerate}
\subsection{Video and Website Content}
\subsubsection{Goals}
\begin{enumerate}
\item Keep 101 and 110 website up to date with lecture materials. 
\item Produce marketing videos for 110/101 events 
\item Produce educational videos of 110 and 101 materials
\end{enumerate}
\subsubsection{Roles}
\begin{enumerate}
\item Video Producer-3 Hour Commitment
\item 3 Video Production Assistants-1 Hour Commitment 
\item Head of COMP110 web content-1 Hour Commitment
\item Head of COMP101 web content-1 Hour Commitment
\item 2 website content creator-1 Hour Commitment
\item Website content editor-1 Hour Commitment
\end{enumerate}
\subsection{Worksheets}
\subsubsection{Goals}
\begin{enumerate}
\item Create biweekly worksheets for COMP110
\item Create optional worksheets for extra COMP110 practice
\item Create optional worksheets for COMP101
\end{enumerate}
\subsubsection{Roles}
\begin{enumerate}
\item Head of 110 Worksheets-2 Hour Commitment
\item Head of 101 Worksheets-1 Hour Commitment
\item 4 Problem Creators-1 Hour Commitment 
\end{enumerate}
\section{Procedures}
\subsection{Schedule Generation}
\begin{itemize}
\item \textbf{Sunday}
Remind everyone at all hands or afterwards on Slack that this is a schedule generation week. All schedule updates need to be in by Wednesday at 5pm.
\item \textbf{Wednesday}
At 5pm put out a last call for schedule updates
\end{itemize}
\subsection{\label{exam_grading}Gradescope post exam}
\subsubsection{Feeder} – You are cutting corners of exam packs and placing them in the scanner
\\
\begin{itemize}
\item Check the version of the exam. Stay with one version until it is completed. Never scan two versions in the same scan!
\item When changing versions, notify the "scanner"
\item Move scratch paper to back of each exam.
\item Cut off corners of 3 exams at a time.
\item Place paper face down, pointing downward, in the scanner.
\item Let the "Scanner" role press the blue button. When the scan completes, take the pages out and place them face down into a new stack that gets handed back to the "Staplerer" role.
\end{itemize}

\subsubsection{Scanner} – You are scanning exams and uploading them to gradescope
Press the blue button when ready to scan.
\begin{itemize}
\item If scan goes wrong in anyway, do not proceed until resolved that all exams are scanned correctly.
\item Save scan to correct folder on desktop (organized by exam version)
Be sure you are on the right version in gradescope! Esp when versions change!
Go to manage scans in Gradescope sidebar. Drag saved PDF to the upload scans area.
\item Click "show" to see submissions.
\item Review each of the suggested submissions to ensure that they are actually correct. They commonly are not if back pages were ripped off and thrown away.
Create submissions.
\end{itemize}
\subsubsection{Staplerer} – You are restapling scanned exams

\begin{itemize}
\item Break stack of scans back into individual packets 
\item Staple
\item Stack back up again, face down, in alternating stack of perpendicularity – this ensures exams remain in the same order in which they were scanned
\end{itemize}
\subsubsection{Submissioner} – You are assigning students to submissions
\begin{itemize}
\item Take the stack from the staplerer, flip back over
\item Check the version of the exam you are managing the submissions of
\item Go to the "Manage Submissions" tab of that exam
\item Select the "Unassigned" tab to show submissions unassigned
\item Take the first exam off the top of your current stack. It should be the next unassigned exam in gradescope. Check to be sure these match. If they do not, it means one of two things:
\begin{itemize}
\item Either something has gone wrong in the process above. Make sure everyone is stacking correctly. Check the bottom of the stack to be sure a previous part of the process did not accidentally reverse the order of the stack.
\item OR this exam accidentally did not get scanned. This is when we want to catch this error and why we are stacking so deliberately. Be sure this exam is scanned again.
\end{itemize}
\end{itemize}
\subsection{Shift leads}
\textbf{General Purpose of Role:} \\
Ensure office hours operate smoothly and maintain a high quality of service for our students.\\
\\
\textbf{Before your shift}
\begin{itemize}
\item Check Shift Overflow and verify you are the Shift Lead before your shift starts. 
\begin{itemize}
\item As people swap in and out of shifts someone with a higher “rank” may have superseded you as the Shift Lead. This “rank” is based on semesters served with the team and not finely tuned beyond that.
\end{itemize}
\end{itemize}
\textbf{At the beginning of your shift}
\begin{itemize}
\item It is imperative as the Shift Lead you arrive on time if not a minute or two early.
\item Check in with the Shift Lead for the previous hour to see if there are common student issues or any other topics that need to be addressed.
\item Look around office hours and compare who is there with who is on the schedule.
\begin{itemize}
\item If it is 5 minutes past the hour, message any missing TAs on Slack to check-in.
\end{itemize}
\end{itemize}
\textbf{During your shift}
\begin{itemize}
\item If there is no queue in office hours, let others call tickets and remain available as a “floater”. TAs can come to you if they run into technical problems they are get stuck on.
\item Keep an eye on the general flow of office hours and have a sense of what are the common problems students come in with.
\item Assign TAs to monitor either the 110 queue or the 101 queue. Use your intuition to determine what the distribution should be and rebalance as necessary to achieve the best, most fair level of service we can to students looking for help in both courses. As an example, if 110 has a steady 10 people in it and 101 has a steady 2 people in it ... and we have 5 TAs ... I would distribute 4 to 110 and 1 to 101.
\item You should not hover over others team members’ shoulders unless asked for help or mentorship directly.
\item When office hours are slow: 
\begin{itemize}
\item Ensure tickets are being distributed equally across the TAs as they come into the queue. There should not be a single TA calling all of the tickets.
\item If there is grading to be done in Gradescope, please ensure everyone pitches in to make progress during your shift.
\item Walk around and chat with the other TAs. Use this as a time to get to know the team and provide mentorship (check in on how things are going, give advice on classes, etc.).
\end{itemize}
\item When office hours are busy:
\begin{itemize}
\item \textbf{Do} take tickets! However, after you complete a ticket, size up the current status of office hours and the queue. Perhaps take a quick stroll around the room and see if anyone grabs you in need of help.
\item Keep an eye on how long fellow team members are spending with students when there is a queue. Keep an eye on appointments > 15 minutes and, if necessary, step in to help bring it to a close.
\end{itemize}
\textbf{At the end of your shift}
\item Fill out the following form at the end of each hour to record any concerns encountered during your shift: \href{https://goo.gl/forms/GtdjeZyl3ZPxfp8N2}{https://goo.gl/forms/GtdjeZyl3ZPxfp8N2}
\item Meet with the incoming shift lead to discuss any important notes for the upcoming hour.
\end{itemize}

\subsection{\label{hour_allocation} Adjusting hour allotment}
\begin{itemize}
\item Zero out column for in class help on \textit{Capacity Allocation Spreadsheet}
\item Update after setting up a new in-class help schedule
\begin{itemize}
\item M/W = 2 hours
\item T/R = 3 hours
\item F = 1 hour
\end{itemize}
\item Update Shift Overflow with new Hour allotments  - go through ALL TAs with their Onyen from the spreadsheet and put in the OH number listed in the column after updating and make sure they are not marked available for their hours! See below for how to update hours – also manually check if they are marked as available when doing in class lecture. 
\end{itemize}
\subsection{\label{team_email}Checking Team Emails}
\begin{itemize}
\item Emails is a task that is have set to repeat every 3 days on Asana, once you check through emails just mark it as complete and it will reset. Typically 3 days is a solid amount of time but feel free to adjust as you need.
\item If you are now in charge of email, have Kris add you as an administrator in the Google groups. 
\item On the back end all emails in gray have been looked at -> since you are new all will appear white, go ahead and click mark all as read. 
\item Once you start regularly checking, if you come across an email that has been unanswered (within 24 hours or not) mark as unanswered. The ones that are more than 24 hours, send a nudge on slack to the people. 
\end{itemize}
\section{Outreach}
\subsection{\label{quiz_outreach}Quizzes and Exams}
\begin{itemize}
\item Go into Gradescope <Semester/Course> and select the quiz you want to get the data for.
\item Click review grades in the menu on the left hand side and then download the grades to a CSV using the Download Grades button at the bottom of the page. 
\item Delete all columns from the CSV except name, email, and total scores. 
\item Sort the names by column from going into data clicking the A to Z sort. 
\item Select sort from the sort warning that pops up and repeat the process to combine with other quizzes. 
\item It typically works well to do quiz outreach after 2 quizzes, and get the outreach out the Monday following the second quiz for people who have an average sub 60. 
\end{itemize}
\subsection{Worksheets}
\subsection{Problem Sets}
\section{Diversity Panel}
\section{\label{hackathon}HACK110}
\subsection{Reserving Event Space}
\begin{itemize}
\item Discuss potential dates with Kris as early in semester as possible, semester before if you can. 
\item As soon as you have selected dates meet with Hope to reserve all of lower level Sitterson. 
\begin{itemize}
\item If the hackathon is in the fall, also reserve a lecture hall in Chapman for opening ceremony 
\item 
\end{itemize}
\item Blow up matresses day of and choose a room to put them in 
\end{itemize}
\subsection{Budget}
\begin{itemize}
\item Get estimate of attendance based on which semester and recent attendance
\item Get prices based on Sam’s club/restaurants for catering
\begin{itemize}
\item It is best to use restaurants to CS department has a relationship with-Hope in the department is a good person to ask about this 
\item Department will also let you use Sam’s club account so you just send them list and they will pre-select and then we just pick up (and maybe grab a few extra items 
\end{itemize}
\item Let Kris see and suggest edits to budget
\item Submit to Hope, Jeffay, and Kris--prod every week or so until approval 
\end{itemize}
\subsection{Swag}
\begin{itemize}
\item Kris handles tshirts--make sure he does not order female cut
\item If you want another swag item remember to budget, would plan for this if fall hackathon, typically larger with more attendance. 
\end{itemize}
\subsection{Food}
\begin{itemize}
\item Overall: Dinner, Snacks, Ice Cream Bar, Snacks, Breakfast
\item Dinner
\begin{itemize}
\item Pizza (Amante’s is usually pretty good)
\item Vegan/Gluten free: Med Deli
\end{itemize}
\item Snacks
\begin{itemize}
\item Granola bars
\item Cookies
\item Chips
\item Vegetable tray
\item Apples 
\end{itemize}
\item Ice Cream bar
\begin{itemize}
\item chocolate
\item vanilla
\item whip cream
\item cherries
\item chocolate sauce
\end{itemize}
\item Breakfast
\begin{itemize}
\item muffins
\item mini bagels
\item cream cheese
\item OJ
\item regular milk
\item non-dairy milk
\item if budge allows donuts
\end{itemize}
\item Drinks
\begin{itemize}
\item Mountain Dew
\item Diet Mountain Dew
\item Dr. Pepper
\item Diet Dr. Pepper
\item water
\item Energy drink (lol probs not redbull)
\end{itemize}
\end{itemize}
\subsection{Organization of Activities}
\begin{itemize}
\item Schedule
\begin{itemize}
\item Overall schedule of intro, workshops, hacking and extra
\end{itemize}
\item TA submitted workshops
\begin{itemize}
\item Have TAs submit ideas and assign them a time slot in schedule
\item Have them send you their ppt about a week before hand
\item TAs can also do late night non coding activities
\end{itemize}
\item Plan non-coding breaks
\begin{itemize}
\item Gaming, cup stacking, escape room?
\end{itemize}
\item Approving projects
\begin{itemize}
\item Have office hours staffed for TAs to approve ideas via course.care
\item Give fake code that they will upload in google form to confirm attendance
\end{itemize} 
\item Project presentation
\begin{itemize}
\item Give students a form to present their project if they want to (encourage them to show off!)
\end{itemize}
\item Scheduling TAs
\begin{itemize}
\item Have schedule of TAs—want to staff heavy early so you can approve projects without a long line
\end{itemize}
\section{Professional Development}
\subsection{Interview Tips}

\subsubsection{Technical}
\begin{itemize}
\item Resources for practice
\begin{itemize}
\item \href{leetcode.com}{leetcode.com}
\item \href{hackerrank.com}{hackerrank.com}
\item \href{https://www.amazon.com/Cracking-Coding-Interview-Programming-Questions/dp/0984782850/ref=pd_lpo_sbs_14_img_0?_encoding=UTF8&psc=1&refRID=FZB5N025M099WJDENG89}{Cracking the Code Interviews(book)}
\end{itemize}
\item Different companies focus on different knowledge in interviews depending on their backend and what the internship focuses on
\item Things to consider when asked a technical question
\begin{itemize}
\item How are you going to test the problem? What kinds of test cases are you going to use? What are the edge cases that you need to account for?
\item What is the time complexity and space complexity of your code and how can you potentially make it more efficient? Efficiency stuff in general is always really important
\end{itemize}
\item Don’t say “I don’t know” say “I don’t know, but…” talk your way around what you don’t know and show that you can think around the issues and problems	
\item Ground your skills in concrete examples (i.e. past projects you have worked on)
\item Be able to explain code/code projects in layman's terms like you do in office hours.
\item Try to know all the data structures that you have learned in class so far, but don’t stress about knowing things that you haven’t had a chance to learn yet.
\item Write/talk out your goal/steps to get to that goal in English.
\item Messy code that works > clean code that doesn't work.
\item Get code to work THEN make time/space efficiency. Start with a naive approach.
\item Ask clarifying questions, don’t make assumptions without confirming/asking.
\item Clarification > speed
\item Know basic calculus
\item Think through problems thoroughly, being sure to voice your thought process aloud and to avoid rushing
\item If you’ve seen a technical question before, tell your interviewer (integrity)
\end{itemize}
\subsubsection{Non-Technical}
\begin{itemize}
\item Not every interview will be coding based - some will be behavioral and more general question - i.e. tell me about your resume, or about a project you worked on and stuff like that
\item Be as personable as you can.
\item Do some research about the companies you’re interested in so you can ask specific questions.
\item Have fun! Learn something along the way. Be someone that others want to work with.
\end{itemize}
\end{itemize}
\subsection{Recruitment}
\begin{itemize}
\item Maintain/Update: github, linkedIn, resume, cover letter, personal website (web.unc.edu)
\item Check intern.supply website 
\item Apply EVERYWHERE and pray
\item Find a good business casual outfit that you feel good in 
\item Keep track of all the places you have applied to and the people/contacts that you have talked to
\item Apply just to get your name out to companies and get used to the interview/application process
\item Reach out to other TAs about specific company’s interview processes
\item If you’re a sophomore, look for the explorer programs
\item Actually dress like you’re going to an interview if you have a phone interview
\item SMILE when you’re talking on the phone - actually makes a huge difference
\item Talk to a few people at the tech fair before going to the big hitters so you can get some practice in
\item Run over your resume at tech fair to impress recruiters
\item Personalize what personal projects you have on resume to the company you are applying to
\end{itemize}
\subsection{Resume}
\begin{itemize}
\item Put school address on resume
\item Make your name big and bold
\item Add your TA stats to your resume
\item Embed your skills within your resume’s job and project descriptions
\item List favorite personal projects and what was interesting about them
\item When describing jobs, write with an emphasis on the verbs of what you uniquely accomplished/practiced, not the nouns and adjectives what the job’s description was.
\item Sample Resume Here
\end{itemize}
\section{The Team 1101 Network}

\subsection{Full Time}
\begin{itemize}
\item Sarah White ('16)-Data Scientist at Asana
\item Dorian Brandon ('17)-Software Engineer at Amazon
\item Karen Cheng ('17)-Software Engineer at ESPN
\item Meggie Cruser ('17)-Consultant at Accenture
\item Krista Bellamy ('18)-Software Engineer at IBM
\item Melissa Fu ('18)-Technology Analyst at JP Morgan
\end{itemize}
\subsection{Explorer-Type Internships}
\begin{itemize}
\item Helen Qin-Google Engineering Practicum Intern 2017
\item Mary Gibeau-Microsoft Explorers Intern 2018
\end{itemize}
\subsection{General Internships}
\subsubsection{2017}
\begin{itemize}
\item Tabatha Seawell-Software Engineer, Bivarus
\item Brooks Townsend-Software Development, SentryOne
\item Kyra Mulder-Software Engineer, Fidelity Investments
\item Heather Crew-Software Engineer, Fidelity Investments
\end{itemize}
\subsubsection{2018}
\begin{itemize}
\item Helen Qin- Software Engineer, Google 
\item Kate Goldenring-Product Manager, Microsoft 
\item Hank Hester-Software Engineer, Microsoft
\item Brooks Townsend-Software Engineer, CapitalOne 
\item Kiet Huynh-Software Engineer, Amazon 
\item Sarah Ganci-Product Management, GE 
\item Gabi Stein-Product Analyst, Quizlet 
\item Morgan Vickery-Software Engineer, WillowTree 
\item Sydney Cole-Software Engineer, Goldman Sachs 
\item Rhett Dudley-Software Engineer, SAS 
\item Izzi Hinks-Sustainability Analyst, SAS
\item Carly Clark-Software Engineer, Fidelity Investments 
\item Kristina Nickel-Systems Engineer, Fidelity Investments
\item Tiara Mathur-Software Engineer, Fidelity Investments
\item Zak Lintz-Software Engineer, SentryOne
\end{itemize}
\end{document}
\stopcontents[mainsections]